\chapter*{Abstract}
\markright{Abstract}

The reversible object-oriented programming language (\textsc{Roopl}) was presented in late 2016 and proved that object-oriented programming paradigms was worked in the reversible setting. The language featured simple statically scoped objects which made non-trivial programs tedious, if not impossible to write using the limited tools provided.
We introduce an extension to \textsc{Roopl} in form the new language \rooplpp, featuring dynamic memory management and static arrays for increased language expressiveness. The language is a superset of \textsc{Roopl} and has formally been defined in its language semantics, type system and computational universality. Considerations for reversible memory manager layouts were discussed and ultimately led to the selection of the Buddy Memory layout. Translations of the extensions added in \rooplpp to the reversible assembly language \textsc{Pisa} are presented to provide garbage-free computations. The dynamic memory management extension successfully increased the expressiveness of \textsc{Roopl} and as a result, showed that non-trivial reversible data structures, such as reversible binary trees and doubly-linked lists, are feasible and does not contradict the reversible computing paradigm.